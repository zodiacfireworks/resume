%!TEX TS-program = xelatex
%!TEX encoding = UTF-8 Unicode
% Preambulo --------------------------------------------------------------------
\documentclass[11pt, a4paper]{awesome-cv}
\usepackage{polyglossia}
\usepackage{getlang}

\ifLang{\esLang}{\setmainlanguage{spanish}}
\ifLang{\enLang}{\setmainlanguage{english}}
\fontdir[fonts/]
\definecolor{red-softbutterfly}{rgb}{0.7843137254901961, 0.21568627450980393, 0.21568627450980393}
\colorlet{awesome}{red-softbutterfly}
%\colorlet{awesome}{awesome-red}

\definecolor{darktext}{HTML}{424242}
\definecolor{text}{HTML}{212121}
\definecolor{graytext}{HTML}{616161}
\definecolor{lighttext}{HTML}{bdbdbd}

% Set false if you don't want to highlight section with awesome color
\setbool{acvSectionColorHighlight}{true}

% If you would like to change the social information separator from a pipe (|) to something else
\renewcommand{\acvHeaderSocialSep}{\quad\textbar\quad}
\newcommand{\present}{
  \ifLang{\esLang}{PRESENTE}%
  \ifLang{\enLang}{PRESENT}%
}
\newcommand{\translate}[2]{
  \ifLang{\esLang}{#1}%
  \ifLang{\enLang}{#2}%
}

%-------------------------------------------------------------------------------
% PERSONAL INFORMATION
% Comment any of the lines below if they are not required
%-------------------------------------------------------------------------------
% Available options: circle|rectangle,edge3/noedge,left/right
% \photo[rectangle,edge,right]{./examples/profile}
\name{Martín}{Vuelta}
\position{
  \translate{
    Especialista en observabilidad | Especialista cloud | Desarrollador backend
  }{
    Observability Specialist | Cloud Specialist | Backend Developer
  }
}
\address{Lima, \translate{Perú}{Peru}}

\mobile{(+51) 982042088}
\email{martin.vuelta@gmail.com}
\github{zodiacfireworks}
\linkedin{martinvuelta}
% \gitlab{gitlab-id}
% \stackoverflow{SO-id}{SO-name}
% \twitter{@twit}
% \skype{skype-id}
% \reddit{reddit-id}
% \extrainfo{extra informations}
% \quote{``Sé el cambio que deseas ver en el mundo."}

%-------------------------------------------------------------------------------
\begin{document}

% Print the header with above personal informations
% Give optional argument to change alignment(C: center, L: left, R: right)
\makecvheader[C]

% Print the footer with 3 arguments(<left>, <center>, <right>)
% Leave any of these blank if they are not needed
\makecvfooter
{\today}
{
  Martín Vuelta
  ~~~·~~~
  \translate{Currículo}{CV}
}
{\thepage}


%-------------------------------------------------------------------------------
% CV/RESUME CONTENT
% Each section is imported separately, open each file in turn to modify content
%-------------------------------------------------------------------------------
%-------------------------------------------------------------------------------
%	SECTION TITLE
%-------------------------------------------------------------------------------
\translate{\cvsection{Resumen}}{\cvsection{Summary}}

%-------------------------------------------------------------------------------
%	CONTENT
%-------------------------------------------------------------------------------
\begin{cvparagraph}
    \translate{
        Cofundador y director de Investigación y Desarrollo de Software en SoftButterfly. Con más de 5 años de experiencia en desarrollo de software y 3 años de experiencia en seguridad informática e infraestructura \em{cloud}.

        Me gustan las ciencias y siempre estoy buscando la mejor manera de resolver problemas teóricos o experimentales mediante la aplicación de la informática para lograr una mejor comprensión del mundo.
    }{
        Co-founder and director of Software Research and Development at SoftButterfly. With more than 5 years of experience in software development and 3 years of experience in both computer security and cloud infrastructure.

        I like science and I am always looking for the best way to solve theoretical or experimental problems by applying information technology to achieve a better understanding of the world.
    }
\end{cvparagraph}

\noindent

% \begin{minipage}[t]{.65\textwidth}
% \end{minipage}\hfill% <---------------- Note the use of "%"
% \begin{minipage}[t]{.325\textwidth}
% \end{minipage}

%	SECTION TITLE ----------------------------------------------------------------
\translate{\cvsection{Habilidades}}{\cvsection{Skills}}

%	CONTENT ----------------------------------------------------------------------
\begin{cvskills}

  %----------
  \cvskill
    {
      \translate{Programación}{Programming}
    } % Category
    {Python, C, C++, C\#, Fortran, Matlab, Mathematica, JavaScript, LaTeX, Java, Julia, Kotlin} % Skills

  %----------
  \cvskill
  {
    \translate{Librerías}{Libraries}
  } % Category
  {LAPACK, BLAS, Intel MKL, ILnumerics, OpenCV, Cuda, OpenGL, SFML} % Skills

  %----------
  \cvskill
  {
    \translate{Hardware}{Hardware}
  } % Category
  {Arduino, Raspberry Pi} % Skills

  %----------
  \cvskill
    {
      \translate{Móvil}{Mobile}
    } % Category
    {Android} % Skills

  %----------
  \cvskill
    {
      \translate{Web}{Web}
    } % Category
    {Wagtail, Django, Flask, Express, HTML5, CSS3, SASS, Gulp} % Skills

  %----------
  \cvskill
    {
      \translate{Base de Datos}{Databases}
    } % Category
    {MySql, MariaDB, PostgreSql, Redis, SQLite} % Skills

  %----------
  \cvskill
    {
      \translate{Cloud}{Cloud}
    } % Category
    {Amazon Web Services, Google Cloud Platform, IBM Bluemix} % Skills

  %----------
  \cvskill
  {
    \translate{Idiomas}{Languages}
  } % Category
  {Español, Inglés} % Skills

%---------------------------------------------------------
\end{cvskills}

%-------------------------------------------------------------------------------
% SECTION TITLE
%-------------------------------------------------------------------------------
\translate{\cvsection{Experiencia}}{\cvsection{Experience}}

%-------------------------------------------------------------------------------
% CONTENT
%-------------------------------------------------------------------------------
\begin{cventries}
  %-----------------------------------------------------------------------------
  \cventry
  {
    \translate{
      Cofundador y Director de Desarrollo e Investigación Tecnologica
    }{
      Co-founder and Director of Development and Technology Research
    }
  } % Job title
  {SoftButterfly} % Organization
  {Lima, Peru} % Location
  {
    Feb. 2017 - \present
  } % Date(s)
  {
    \begin{cvitems} % Description(s) of tasks/responsibilities
      \item {
        \translate{
          Desarrollo de los proyectos Cabildos Bicentenarios y El País Que
          Queremos, con Django, Django RestFramework, Wagtail, React, Tailwind,
          Sass, PostgreSQL y Redis.\newline
          Por encargo del Ministerio de Cultura del Perú en colaboración with HackSpace y Crehana.
        }{
          Development of projects Cabildos Bicentenarios and El País Que
          Queremos, with Django, Django RestFramework, Wagtail, React, Tailwind,
          Sass, PostgreSQL and Redis.\newline
          Commissioned by the Ministry of Culture of Peru In collaboration with HackSpace and Crehana.
        }\newline
        \url{https://bicentenario.gob.pe/cabildos/}\newline
        \url{https://bicentenario.gob.pe/elpaisquequeremos/}
      }
      \item {
        \translate{
          Eigensolver variacional cuántico para la determinación de la energía
          del estado fundamental en moléculas pequeñas.
        }{
          Quantum Variational Eigensolver for ground state energy determination
          in small molecules.
        }\newline
        \url{https://github.com/zodiacfireworks/qiskit-vqe-benchmarcking}
      }
      \item {
        \translate{
          Diseño y desarrollo de un sistema de diagnostico, registro y
          seguimiento de pruebas de COVID 19 para la clínica Repromedic.
        }{
          Design and development of a system for the diagnosis, registration
          and monitoring of COVID-19 tests for the Repromedic clinic.
        }
      }
      \item {
        \translate{
          Desarrollo de un eCommerce con Django, Django RestFramework, VueJS,
          PostgreSQL and Redis para Quimder.
        }{
          Development of an eCommerce with Django, Django RestFramework, VueJS,
          PostgreSQL and Redis for Quimder.
        }\newline
        \url{https://tienda.quimder.com/}
      }
      \item {
        \translate{
          Investigación y desarrollo de herramientas para la migracion de
          tableros de control de DataDog a New Relic y su empleo en New Relic One.
        }{
          Research and development of tools for the migration of dashboards
          from DataDog to New Relic and its use in New Relic One.
        }
      }
      \item {
        \translate{
          Instalacion de agentes de New Relic APM e Infrastructure en conjunto
          con Atentus para los sistemas de Interbank.
        }{
          Installation of New Relic APM and Infrastructure agents in
          conjunction with Atentus for Interbank systems.
        }
      }
      \item {
        \translate{
          Instrumentación de New Relic en conjunto con Atentus para la web de
          BPI de Interbank en VueJs.
        }{
          New Relic instrumentation in conjunction with Atentus for the
          Interbank BPI website in VueJs.
        }
      }
      \item {
        \translate{
          Implementación de New Relic en conjunto con Atentus para los sistemas
          de gestión internos de Sura.
        }{
          New Relic implementation in conjunction with Atentus for Sura's
          internal management systems.
        }
      }
      \item {
        \translate{
          Desarrollo de un sistema de \textit{eLearning} usando Django,
          Wagtail, Bootstrap and PostgresSQL.
        }{
          Development of and \textit{eLearning} platform with Django, Wagtail,
          Bootstrap and PostgresSQL.
        }
      }
      \item {
        \translate{
          Desarrollo de un sistema de registro de preguntas y generación de
          evaluaciones con Django, Django RestFramework, VueJS and PostgreSQL.
        }{
          Development of a survey generator and alalysis system with Django,
          Django RestFramework, VueJS and PostgreSQL.
        }
      }
      \item {
        \translate{
          Implementación de infraestructura de los sistemas internos de
          SoftButterfly empleando varias herramientas del \textit{stack} de
          AWS.
        }{
          Infrastructure implementation of internal SoftButterfly systems using
          various tools from the AWS stack.
        }
      }
      \item {
        \translate{
          Investigación y desarrollo de estaciones meteorológicas empleando
          \textit{comodity hardware} y un sistema personalizado de IoT con
          Django, Django RestFramework, Bootstrap and PostgreSQL.
        }{
          Research and development of meteorological stations using commodity
          hardware and a customized IoT system with Django, Django
          RestFramework, Bootstrap and PostgreSQL.
        }
      }
      \item {
        \translate{
          Investigación para el desarrollo de sistema de seguimiento y
          recuperación de globos meteorológicos.
        }{
          Research for the development of a monitoring and recovery system for
          meteorological balloons.
        }
      }
      \item {
        \translate{
          Investigación y desarrollo de sistema de caracterización de mosquitos
          por la frecuencia de su aleteo.
        }{
          Research for the development of a mosquito characterization system by
          the frequency of its flutter.
        }
      }
    \end{cvitems}
  }

  %-----------------------------------------------------------------------------
  \cventry
    {\translate{Cosultor externo de Fortran}{External Fortran Consultant}} % Job title
    {
      \translate{
        Servicio nacional de Meteorología e hidrología del Perú
      }{
        National Service of Meteorology and Hydrology of Perú
      }
      -
      Senamhi
    } % Organization
    {Lima, Peru} % Location
    {Nov. 2017 - May. 2018} % Date(s)
    {
      \begin{cvitems} % Description(s) of tasks/responsibilities
        \item {
          \translate{
            Capacitación en Fortran
          }{
            Fortran Training
          }
        }
        \item {
          \translate{
            Desarrollo de programas de análisis de datos.
          }{
            Development of data analysis programs.
          }
        }
      \end{cvitems}
    }

  %-----------------------------------------------------------------------------
  \cventry
    {Lead Trainer} % Job title
    {HackSpace Perú} % Organization
    {Lima, Perú} % Location
    {\translate{Ene}{Jan}. 2015 - \translate{Dic}{Dec}. 2019} % Date(s)
    {
      \begin{cvitems} % Description(s) of tasks/responsibilities
        \item {
          \translate{
            Entrenador de desarrollo web full stack en el PADT 2017 (Programa
            de Alto Desempeño Tecnológico) en colaboración con el Ministerio
            de la Producción del Perú.
          }{
            Full stack web development coach at the PADT 2017 (High
            Technological Performance Program) in collaboration with the
            Ministry of Production of Peru.
          }
        }
        \item {
          \translate{
            Entrenador de IoT de los entrenamientos presenciales del
            CoreUpgrade 2017.
          }{
            IoT coach of the CoreUpgrade 2017 face-to-face training.
          }
        }
        \item {
          \translate{
            Entrenador de desarrollo backend de los entrenamientos presenciales
            del CoreUpgrade 2016.
          }{
            Backend development coach of the CoreUpgrade 2016 face-to-face
            training.
          }
        }
      \end{cvitems}
    }

  %-----------------------------------------------------------------------------
  \cventry
    {Lead Software Developer} % Job title
    {HackSpace Perú} % Organization
    {Lima, Perú} % Location
    {Mar. 2016 - PRESENTE} % Date(s)
    {
      \begin{cvitems} % Description(s) of tasks/responsibilities
        \item {
          \translate{
            Planeamiento, diseño,  desarrollo e implementación de toda la
            infraestructura web de HackSpace Perú.
          }{
            Planning, design, development and implementation of the entire web
            infrastructure of HackSpace Peru.
          }
        }
        \item {
          \translate{
            Implementación de sistema de e-Learning para el programa de
            entrenamientos anuales CoreUpgrade 2017.
          }{
            Implementation of e-Learning system for the CoreUpgrade 2017 annual
            training program.
          }
        }
      \end{cvitems}
    }

  %-----------------------------------------------------------------------------
  \cventry
    {Web Developer Freelance} % Job title
    {Grupo RPP} % Organization
    {Lima, Perú} % Location
    {Ago. 2017 - Sep. 2017} % Date(s)
    {
      \begin{cvitems} % Description(s) of tasks/responsibilities
        \item {
          \translate{
            Desarrollo de visualizaciones web interactivas empleando GreenSock,
            D3JS y HighCharts.
          }{
            Development of interactive web visualizations using GreenSock,
            D3JS and HighCharts.
          }
        }
        \begin{itemize}
          \item {\href{https://goo.gl/oCo4gW}{Las cifras del caos}.}
          \item {\href{https://goo.gl/xPWHUZ}{El viaje de la papeleta.}.}
          \item {\href{https://goo.gl/KPA5DA}{Las victimas mortales}.}
          \item {\href{https://goo.gl/jwA8Kh}{Cinco años de tragedias}.}
        \end{itemize}
      \end{cvitems}
    }

  %-----------------------------------------------------------------------------
  \cventry
    {
      \translate{
        Consultor de Seguridad Informática y Desarrollo Web con Python
      }{
        IT Security and Web Development Consultant with Python
      }
    } % Job title
    {DGT Digital} % Organization
    {Lima, Perú} % Location
    {Feb. 2017 - Mar. 2017} % Date(s)
    {
      \begin{cvitems} % Description(s) of tasks/responsibilities
        \item {
          \translate{
            Capacitación de desarrollo web backend con Python y Django.
          }{
            Web development training backend with Python and Django.
          }
        }
        \item {
          \translate{
            Capacitación de Seguridad Informática siguiendo el Estándar OWASP
            para mejorar la seguridad de sis sistemas informáticos.
          }{
            Computer Security Training following the OWAS Standard to improve
            the security of your computer systems.
          }
        }
        \item {
          \translate{
            Capacitación de uso y configuración de Nginx y SSL.
          }{
            Training of use and configuration of Nginx and SSL.
          }
        }
      \end{cvitems}
    }

  %-----------------------------------------------------------------------------
  \cventry
    {Dynamics NAV Developer} % Job title
    {UnionLabelNet} % Organization
    {Lima, Perú} % Location
    {\translate{Dec}{Dic}. 2016 - Mar. 2017} % Date(s)
    {
      \begin{cvitems} % Description(s) of tasks/responsibilities
        \item {
          \translate{
            Desarrollador de sistemas ERP con Microsoft Dynamics NAV.
          }{
            ERP system developer with Microsoft Dynamics NAV.
          }
        }
        \item {
          \translate{
            Desarrollo del \href{https://goo.gl/4iA8ma}{\textit{plugin} C-AL}
            de VSCode para la mejorar la eficiencia en la programación en C-AL
            de Microsoft Dynamcis NAV.
          }{
            Development of the VSCode \href{https://goo.gl/4iA8ma}{C-AL plugin}
            to improve the efficiency in Microsoft Dynamcis NAV C-AL
            programming.
          }
        }
      \end{cvitems}
    }

  %-----------------------------------------------------------------------------
  \cventry
    {Software Developer} % Job title
    {HackSpace Perú} % Organization
    {Lima, Perú} % Location
    {Nov. 2015 - Nov. 2016} % Date(s)
    {
      \begin{cvitems} % Description(s) of tasks/responsibilities
        \item {
          \translate{
            Desarrollo \textit{frontend} de los servicios web de HackSapce.
          }{
            Frontend development of HackSapce web services.
          }
        }
        \begin{itemize}
          \item {
            \translate{
              Web promocional para el Entrenamiento de Comunidades 2015
            }{
              Promotional Website for Community Training 2015
            }
          }
          \item {
            \translate{
              Web promocional para el CoreUpgrade 2016
            }{
              Promotional Website for the CoreUpgrade 2016
            }
          }
        \end{itemize}
      \end{cvitems}
    }

  %-----------------------------------------------------------------------------
  \cventry
    {
      \translate{
        Asistente de investigación en prácticas
      }{
        Research assistant trainee
      }
    } % Job title
    {
      \translate{
        Comisión Nacional de Investigación y Desarrollo Aeroespacial -
        Agencia espacial del Perú
      }{
        National Aerospace Research and Development Commission -
        Peru Space Agency
      }
    } % Organization
    {Lima, Perú} % Location
    {\translate{Abr}{Apr}. 2014 - \translate{Dic}{Dec}. 2015} % Date(s)
    {
      \begin{cvitems} % Description(s) of tasks/responsibilities
        \item {
          \translate{
            Investigación sobre algoritmos para la  predicción de la actividad
            solar.
          }
          {
            Research on algorithms for solar activity forecasting.
          }
        }
        \item {
          \translate{
            Desarrollo de sistemas de monitoreo meteorológico.
          }
          {
            Development of meteorological monitoring systems.
          }
        }
        \item {
          \translate{
            Automatización calculo de efemérides.
          }
          {
            Automation calculation of ephemeris.
          }
        }
      \end{cvitems}
    }

  %-----------------------------------------------------------------------------
  \cventry
    {Software Developer} % Job title
    {DJEngineering} % Organization
    {Lima, Perú} % Location
    {Jul. 2015 - Jul. 2016} % Date(s)
    {
      \begin{cvitems} % Description(s) of tasks/responsibilities
        \item {
          \translate{
            Soporte y desarrollo de características en \textit{backend} y
            \textit{frontend} de sistemas web.
          }{
            Support and development of features in backend and frontend of web
            systems.
          }
        }
      \end{cvitems}
    }

  %-----------------------------------------------------------------------------
  % \cventry
  %   {
  %     \translate{
  %       Técnico en metrología
  %     }{
  %       Metrology Technician
  %     }
  %   } % Job title
  %   {Metrología y Técnicas CP} % Organization
  %   {Lima, Perú} % Location
  %   {Jul. 2013 - \translate{Dec}{Dic}. 2013} % Date(s)
  %   {
  %     \begin{cvitems} % Description(s) of tasks/responsibilities
  %       \item {
  %         \translate{
  %           Técnico en metrología para la calibración de equipos de medición
  %           de masa, longitud, tiempo, volumen, temperatura, humedad y fuerza.
  %         }{
  %           Metrology technician for the calibration of equipment for
  %           measuring mass, length, time, volume, temperature, humidity and
  %           force.
  %         }
  %       }
  %       \item {
  %         \translate{
  %           Automatización de los análisis de datos de calibración usando
  %           Matlab.
  %         }{
  %           Automation of calibration data analysis using Matlab.
  %         }
  %       }
  %       \item {
  %         \translate{
  %           Elaboración de guiones de medición y plantilla de informe de
  %           calibración de tamices siguiendo las especificaciones de la NTC
  %           32 (Norma Técnica Colombiana)
  %         }{
  %           Preparation of measurement scripts and sieve calibration report
  %           template following the specifications of NTC 32 (Colombian
  %           Technical Standard)
  %         }
  %       }
  %     \end{cvitems}
  %   }
  %-----------------------------------------------------------------------------
  % \cventry
  %   {\translate{Docente}{Teacher}} % Job title
  %   {Pro Ciencia} % Organization
  %   {Lima, Perú} % Location
  %   {Mar. 2013 - Jun. 2013} % Date(s)
  %   {
  %     \begin{cvitems} % Description(s) of tasks/responsibilities
  %       \item {
  %         \translate{
  %           Docente del curso de Química de nivel secundaria y primaria.
  %         }{
  %           Teacher of the Chemistry course of secondary and primary level.
  %         }
  %       }
  %     \end{cvitems}
  %   }
  %-----------------------------------------------------------------------------
  % \cventry
  %   {\translate{Docente}{Teacher}} % Job title
  %   {TRENTO} % Organization
  %   {Lima, Perú} % Location
  %   {Jun. 2012 - \translate{Dec}{Dic}. 2012} % Date(s)
  %   {
  %     \begin{cvitems} % Description(s) of tasks/responsibilities
  %       \item {
  %         \translate{
  %           Docente de Física y Química nivel de secundaria.
  %         }{
  %           Physics and Chemistry teacher high school level.
  %         }
  %       }
  %     \end{cvitems}
  %   }
  %-----------------------------------------------------------------------------
  % \cventry
  %   {\translate{Docente}{Teacher}} % Job title
  %   {TRENTO} % Organization
  %   {Lima, Perú} % Location
  %   {Abr. 2009 - Ago. 2009} % Date(s)
  %   {
  %     \begin{cvitems} % Description(s) of tasks/responsibilities
  %       \item {
  %         \translate{
  %           Asesor académico de los cursos de Aritmética, Álgebra, Geometría,
  %           Trigonometría, Razonamiento Matemático, Física y Química de nivel
  %           secundaria.
  %         }{
  %           Academic advisor of the courses of Arithmetic, Algebra, Geometry,
  %           Trigonometry, Mathematical Reasoning, Physics and Chemistry of
  %           secondary level.
  %         }
  %       }
  %     \end{cvitems}
  %   }
  %-----------------------------------------------------------------------------
\end{cventries}

%-------------------------------------------------------------------------------
%	SECTION TITLE
%-------------------------------------------------------------------------------
\translate{\cvsection{Educación}}{\cvsection{Studies}}


%-------------------------------------------------------------------------------
%	CONTENT
%-------------------------------------------------------------------------------
\begin{cventries}

%---------------------------------------------------------
  \cventry
    {B. S. en Física} % Degree
    {Universidad Nacional Mayor de San Marcos} % Institution
    {Lima, Perú} % Location
    {
      Mar. 2008 - \present
    } % Date(s)
    {
      % \begin{cvitems} % Description(s) bullet points
      %   \item {Participé de la }
      % \end{cvitems}
    }
%---------------------------------------------------------
\end{cventries}

% SECTION TITLE ----------------------------------------------------------------
\translate{
  \cvsection{Certificaciones}
}{
  \cvsection{Certifications}
}

% CONTENT ----------------------------------------------------------------------
\begin{cventries}
  \cventry
{
  Dynatrace
}
{
  Dynatrace Partner Sales Specialist
}
{}
{
  Feb. 2025
}
{
  \begin{cvitems}
    \item {
                \href
                {https://www.credly.com/badges/574a67c7-3730-4197-bb85-8dfe9c7de68b}
                {Certificate of Quantum Excellence \faIcon{external-link-alt}}
          }
  \end{cvitems}
}

  \cventry
{
  Dynatrace
}
{
  Dynatrace Partner Sales Certification
}
{}
{
  Feb. 2025
}
{
  \begin{cvitems}
    \item {
                \href
                {https://www.credly.com/badges/b3bf01ba-0bd1-45f2-862f-6aaa6166fd21}
                {Certificate of Quantum Excellence \faIcon{external-link-alt}}
          }
  \end{cvitems}
}

  \cventry
{
  IBM
}
{
  IBM Quantum Challenge 2024 Achievement
}
{}
{
  Jun. 2024
}
{
  \begin{cvitems}
    \item {
                \href
                {https://www.credly.com/badges/3a94706b-4a21-4647-a044-37b0b40fd795/linked_in_profile}
                {Certificate of Quantum Excellence \faIcon{external-link-alt}}
          }
  \end{cvitems}
}

  \cventry
{
  New Relic
}
{
  Partner - Sales Enabled
}
{}
{
  May. 2024
}
{
  \begin{cvitems}
    \item {
                Credential ID: 105099829
          }
    \item {
                \href
                {https://credentials.newrelic.com/f7a0d276-5f6f-47db-bf6f-13193ecb2fd3}
                {Certificado \faIcon{external-link-alt}}
          }
  \end{cvitems}
}

  \cventry
{
  New Relic
}
{
  Partner - Practitioner Exam
}
{}
{
  May. 2024
}
{
  \begin{cvitems}
    \item {
                Credential ID: 105088583
          }
    \item {
                \href
                {https://credentials.newrelic.com/25175d37-2568-436a-808d-2c35a6eb7992}
                {Certificado \faIcon{external-link-alt}}
          }
  \end{cvitems}
}

  \cventry
{
  FinOps Foundation
}
{
  Introduction to FinOps
}
{}
{
  May. 2024
}
{
  \begin{cvitems}
    \item {
                Credential ID: 6mz39z97hrp4
          }
    \item {
                \href
                {http://verify.skilljar.com/c/6mz39z97hrp4}
                {Certificado \faIcon{external-link-alt}}
          }
  \end{cvitems}
}

  \cventry
{
  IBM
}
{
  IBM Quantum Challenge: Spring 2023 Achievement
}
{}
{
  May. 2023
}
{
  \begin{cvitems}
    \item {
                \href
                {https://www.credly.com/badges/99fcbcae-7a23-47a6-b7b2-f7793f2978e0/linked_in_profile}
                {Certificate of Quantum Excellence \faIcon{external-link-alt}}
          }
  \end{cvitems}
}

  \cventry
{
  IBM
}
{
  Qiskit Global Summer School 2022 - Quantum Excellence
}
{}
{
  Ago. 2022
}
{
  \begin{cvitems}
    \item {
                \href
                {https://www.credly.com/badges/90007340-93a5-47fb-a26e-c0c9e96f5fbe/linked_in_profile}
                {Certificate of Quantum Excellence \faIcon{external-link-alt}}
          }
  \end{cvitems}
}

  \cventry
{
  IBM
}
{
  Qiskit Global Summer School 2021
}
{}
{
  Jul. 2021
}
{
  \begin{cvitems}
    \item {
                \href
                {https://drive.usercontent.google.com/download?id=1YD0EHj1bVJg6FhpOEjxrSWTdHDD_95tB&export=download&authuser=0}
                {Certificate of participation \faIcon{external-link-alt}}
          }
  \end{cvitems}
}

  \cventry
{
  New Relic
}
{
  Full Stack Observability Practitioner Exam
}
{} % Location
{
  Jun. 2021
}
{
  \begin{cvitems}
    \item {
                Credential ID: 33132408
          }
    \item {
                \href
                {https://www.credential.net/5d9ac0cf-ac1c-43e1-9e50-5a76a483c1d2}
                {Certificado \faIcon{external-link-alt}}
          }
  \end{cvitems}
}

  \cventry
{
  Faculdade de Ciências da Universidade do Porto
}
{
  Summer School on Machine Learning and Big Data with Quantum Computing
}
{}
{
    Jul. 2020
}
{}

  \cventry
{
  New Relic
}
{
  APM Fundamentals Certification
}
{}
{
  Sep. 2020
}
{
  \begin{cvitems}
    \item {
                Credential ID: s5a8b7f3pni7
          }
    \item {
                \href
                {https://verify.skilljar.com/c/s5a8b7f3pni7}
                {Certificado \faIcon{external-link-alt}}
          }
  \end{cvitems}
}

  \cventry
{
    IBM
}
{
  Qiskit Global Summer School 2020
}
{}
{
    Jul. 2020
}
{}

  \cventry
{
    Coursera
}
{
    COVID19 Data Analysis Using Python
}
{}
{
    Jun. 2020
}
{
    \begin{cvitems}
        \item {
                    Credential ID: MM2RMT4L83D9
              }
        \item {
                    \href
                    {https://www.coursera.org/account/accomplishments/certificate/MM2RMT4L83D9}
                    {Certificado \faIcon{external-link-alt}}
              }
    \end{cvitems}
}

\end{cventries}

\cvsection{OpenSource}

\begin{cvskills}
  \cvskill
  {
    pypi
  }
  {
    \href
    {https://pypi.org/project/newrelic-sb-sdk/}
    {
      \texttt{newrelic-sb-sdk}
      \faIcon{external-link-alt}
    }
  }

  \cvskill
  {pypi}
  {
    \href
    {https://pypi.org/project/wagtail-sb-imageserializer/}
    {
      \texttt{wagtail-sb-imageserializer}
      \faIcon{external-link-alt}
    }
  }

  \cvskill
  {pypi}
  {
    \href
    {https://pypi.org/project/culqi-api-python/}
    {
      \texttt{culqi-api-python}
      \faIcon{external-link-alt}
    }
  }

  \cvskill
  {npmjs}
  {
    \href
    {https://www.npmjs.com/package/weatherlab}
    {
      \texttt{weatherlab}
      \faIcon{external-link-alt}
    }
  }

\end{cvskills}



%-------------------------------------------------------------------------------
\end{document}
