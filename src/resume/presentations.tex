%-------------------------------------------------------------------------------
%	SECTION TITLE
%-------------------------------------------------------------------------------
\cvsection{Ponencias}

%-------------------------------------------------------------------------------
%	CONTENT
%-------------------------------------------------------------------------------
\begin{cventries}

%---------------------------------------------------------
  \cventry
    {
      \translate{
        I Foro de Tecnologías y Emprendimiento
      }{
        I Forum of Technologies and Entrepreneurship
      }
    } % Event
    {
      \translate{
        Medidas básicas de seguridad para servidores
      }{
        Basic security measures for servers
      }
    } % Speech
    {Universidad Privada del Norte, Lima, Perú} % Location
    {Jun. 2017} % Date(s)
    {
      \begin{cvitems} % Description(s) of tasks/responsibilities
        \item {
          \translate{
            Guias practicas de configuracion de los encabezados de seguridad para servidores web con énfasis en nginx.
          }{
            Practical guides for configuring security headers for web servers with emphasis on nginx.
          }
        }
      \end{cvitems}
    }

  \cventry
    {
      \translate{
        Dia del Código
      }{
        Code Day
      }
    } % Event
    {
      \translate{
        Cellular Automata with Python
      }{
        Cellular Automata with Python
      }
    } % Speech
    {Universidad Privada del Norte, Lima, Perú} % Location
    {May. 2017} % Date(s)
    {
      \begin{cvitems} % Description(s) of tasks/responsibilities
        \item {
          \translate{
            Sobre como profundicé mi interés en los autómatas celulares luego de ver el video "I'm humanity" de Etsuko Yakusimaru. Ejemplos practicos de autómatas celulares realizados en Python.
          }{
            On how I deepened my interest in cellular automata after watching the video "I'm humanity" by Etsuko Yakusimaru. Practical examples of cellular automata made in Python.
          }
        }
        \item {
          \faGithub ~ Github: \url{https://github.com/zodiacfireworks/code-day-2017--cellular-automata}
        }
      \end{cvitems}
    }

  \cventry
    {Flisol Lima} % Event
    {
      \translate{
        Jupyter Notebook: Un entorno de Programación Interactivo
      }{
        Jupyter Notebook: An Interactivo Programming Environment
      }
    } % Speech
    {Universidad Privada del Norte, Lima, Perú} % Location
    {May. 2017} % Date(s)
    {
      \begin{cvitems} % Description(s) of tasks/responsibilities
        \item {
          \translate{
            Acerca del uso de Jupyter notebook en la enseñanza y en entornos avanzados de trabajo real. Ejemplos prácticos en matemática y análisis de datos con Spark.
          }{
            About the use of Jupyter notebook in teaching and in advanced real-work environments. Practical examples in mathematics and data analysis with Spark.
          }
        }
        \item {
          \faGithub ~ Github: \url{https://github.com/zodiacfireworks/talk--jupyter-notebook}
        }
      \end{cvitems}
    }

  \cventry
    {FUDCON: Fedora Users and Developers Conference} % Event
    {Fedora, Python y Arduino} % Speech
    {Puno, Perú} % Location
    {Oct. 2016} % Date(s)
    {
      \begin{cvitems} % Description(s) of tasks/responsibilities
        \item {
          \translate{
            Sobre como construir una aplicación de escritorio con Python y Qt para la recoleccion de datos de diversos sensores conectados a una placa Arduino.
          }{
            On how to build a desktop application with Python and Qt for the collection of data from various sensors connected to an Arduino board.
          }
        }
        \item {
          \faGithub ~ Github: \url{https://github.com/zodiacfireworks/arduino-serial-monitor}
        }
      \end{cvitems}
    }

  \cventry
    {
      \translate{
        XXIII INTERCON Congreso Internacional de Ingeniería, Electrónica, Eléctrica y Computación
      }{
        XXIII INTERCON International Congress of Engineering, Electronics, Electrical and Computing
      }
    } % Speech
    {
      \translate{
        Computación de alto rendimiento con Python
      }{
        High performance computing with Python
      }
    } % Event
    {Universidad de Piura, Piura, Perú} % Location
    {Ago. 2016} % Date(s)
    {
      \begin{cvitems} % Description(s) of tasks/responsibilities
        \item {
          \translate{
            Taller introductorio a la computación de alto rendimiento en Python.
          }{
            Introductory workshop to high performance computing in Python.
          }
        }
        \item {
          \faGithub ~ Github: \url{https://github.com/zodiacfireworks/talk--jupyter-notebook}
        }
      \end{cvitems}
    }

  \cventry
    {Encuentro científico internacional} % Speech
    {Algoritmos computacionales para el análisis de actividad solar.} % Event
    {Piura, Perú} % Location
    {Jul. 2015} % Date(s)
    {}

  \cventry
    {X Conferencia Latinoamericana de Geofísica Espacial} % Speech
    {Solar cycle forecasting.} % Event
    {Cusco, Perú} % Location
    {Sep. 2014} % Date(s)
    {}

  \cventry
    {XI Simposio Peruano de Física y V Congreso Peruano de Física Médica} % Speech
    {Sobre el problema gravitacional inverso en gravedad newtoniana.} % Event
    {Cusco, Perú} % Location
    {Oct. 2012} % Date(s)
    {}

  \cventry
    {XI Simposio Peruano de Física y V Congreso Peruano de Física Médica} % Speech
    {Movimiento de partículas cargadas en campos electromagnéticos.} % Event
    {Cusco, Perú} % Location
    {Oct. 2012} % Date(s)
    {}

  \cventry
    {XV Simposio Nacional de Estudiantes de Física} % Speech
    {Simulaciones Magnetohidrodinámica en física del plasma.} % Event
    {Cusco, Perú} % Location
    {Nov. 2011} % Date(s)
    {}

%---------------------------------------------------------
\end{cventries}
